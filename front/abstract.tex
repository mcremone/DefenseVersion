\thispagestyle{empty}
\begin{raggedleft}
\vspace*{23mm}
\hfill {\huge {\bf {Abstract}}} \\
\vspace{6mm}
\hfill \rule{4in}{.015in} \\
\vspace{19mm}
\end{raggedleft}

\noindent{ }

In this thesis I show how large galaxy surveys, in particular the study of the properties of galaxies, can shed light on gravitational wave sources and dark matter. This is achieved using the latest data from the Dark Energy Survey, an on-going $5000 ~{\rm deg}^2$ optical survey. Galaxy properties such as photometric redshifts and stellar masses are derived through spectral energy distribution fitting methods. The results are used to study host galaxies of gravitational wave events and how light traces dark matter in galaxy clusters.
Gravitational wave (GW) science, and particularly the electromagnetic follow up of these events, is transforming what had never been seen into a new astronomical field able to unveil the nature of cataclysmic events. Identifying the galaxies that host these events, and estimating their redshift, stellar mass, and star--formation rate, is crucial for cosmological analysis with gravitational waves, for follow up studies and to understand the formation of the binary systems that are thought to produce observable gravitational wave signals. This thesis describes how the host matching is implemented within the DES--GW pipeline and how observations of NGC 4993, the galaxy host of the event GW170817, provide important information about possible formation scenarios for binary neutron stars. In particular, we find that NGC 4993 presents shell structures and we relate their formation to the binary formation.
The same galaxy properties are used to derive an observable mass proxy for galaxy clusters. I show that this mass observable correlates well with the total mass of clusters, which is mainly composed of dark matter. It can therefore be used for cosmological studies with galaxy clusters. The measurement of stellar--to--halo mass relations in clusters provides insights on the connection between the star content and the total matter content in clusters, and how this evolves over cosmic time.

%This work aims at unlocking the nature of dark matter and gravitational wave sources, while keeping the connection to an accurate astrophysics of the systems studied by utilizing our knowledge of the astrophysics of galaxies.

%%%%%%%%%%%% Max 300 words

\newpage
\thispagestyle{empty}
\begin{raggedleft}
\vspace*{23mm}
\hfill {\huge {\bf {Impact Statement}}} \\
\vspace{6mm}
\hfill \rule{4in}{.015in} \\
\vspace{19mm}
\end{raggedleft}

\noindent{The analysis presented on host galaxy of the gravitational wave event GW170817 brought evidence that formation mechanisms of the sources of such events differ from what was previously expected. This will have an impact on the theoretical models needed to explain these events. The methodology developed for electromagnetic follow ups and host galaxy analyses will be applied to future observations, and the science results from these studies will affect our current understanding of fields that span from cosmology to particle physics. }

The extensive computational effort required for transient searches with photometric surveys and for photometric redshift (or galaxy properties) estimation, has been an important point of connection with the UCL Center for Doctoral Training (CDT) in Data Intensive Science. I have presented my work at various CDT events, and I hope that in the near future we will strengthen the collaboration between the DIS centre and DES analyses. More generally, developing and testing machine learning codes and methods to handle large datasets affects research inside and outside the world of academia.








