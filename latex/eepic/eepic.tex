\documentstyle[11pt,epic,eepic]{article}
% ALLTT DOCUMENT-STYLE OPTION - released 17 December 1987
%    for LaTeX version 2.09
% Copyright (C) 1987 by Leslie Lamport

% Defines the `alltt' environment, which is like the `verbatim'
% environment except that `\', `\{', and `\}' have their usual meanings.
% Thus, other commands and environments can appear within an `alltt'
% environment.  Here are some things you may want to do in an `alltt'
% environment:
% 
% * Change fonts--e.g., by typing `{\em emphasized text\/}'.
% 
% * Insert text from a file foo.tex by typing `input{foo}'.  Beware that
%   each <return> stars a new line, so if foo.tex ends with a <return>
%   you can wind up with an extra blank line if you're not careful.
% 
% * Insert a math formula.  Note that `$' just produces a dollar sign,
%   so you'll have to type `\(...\)' or `\[...\]'.  Also, `^' and `_'
%   just produce their characters; use `\sp' or `\sb' for super- and
%   subscripts, as in `\(x\sp{2}\)'.
\makeatletter
\def\docspecials{\do\ \do\$\do\&%
  \do\#\do\^\do\^^K\do\_\do\^^A\do\%\do\~}
%
\def\alltt{\trivlist \item[]\if@minipage\else\vskip\parskip\fi
\leftskip\@totalleftmargin\rightskip\z@
\parindent\z@\parfillskip\@flushglue\parskip\z@
\@tempswafalse \def\par{\if@tempswa\hbox{}\fi\@tempswatrue\@@par}
\obeylines \tt \catcode``=13 \@noligs \let\do\@makeother \docspecials
 \frenchspacing\@vobeyspaces}
%
\let\endalltt=\endtrivlist
\makeatother
\ifx\twltt\undefined \font\twltt=cmtt10 scaled \magstep1\fi

\topmargin=-37pt\oddsidemargin=0pt\evensidemargin=0pt
\marginparsep 10pt\marginparwidth 60pt
\textheight 9truein\textwidth 6.5truein

\chardef\BS=`\\ % It only works in \tt font
\newcommand{\PiCTeX}{PiC\TeX}
\newcommand{\epic}{{\sc epic}}
\newcommand{\eepic}{{\sc eepic}}
\newcommand{\plotchar}{\makebox(0,0){\large $\otimes$}}
\parskip=4pt plus 2pt minus 1pt
\title{EEPIC\\Extensions to epic\ and \LaTeX \\ Picture
Environment Version 1.1}
\author{Conrad Kwok\\
  Department of Electrical Engineering and Computer Science\\
  University of California, Davis}
\date{Febrary 2, 1988}
\begin{document}
\maketitle


\section{Introduction}
\LaTeX\ provides a basic but limited picture drawing capability.
\epic\footnote{\epic\ is a \LaTeX\ macro package written by Sunil
Podar at S.U.N.Y at Stony Brook. Please read the section on
installation for more information} is an enhancement to the
picture environment of \LaTeX\ which provides a simpler and more
powerful interface. It introduces new commands for drawing solid
lines, dotted lines, dash lines and new environments suitable for
plotting graphs.

However, \epic\ still inherits many of the limitations of
\LaTeX\ in picture drawing and hence some of the functions either
take a long time to accomplish or the output is not very nice
looking.

tpic is preprocessor program for use with \TeX. It uses a set of
\verb|\special|s graphics commands for drawing pictures. More and
more DVI driver programs supports those specials. They are
becoming a standard set of \verb|\special|s for DVI files.
However, the major disadvantage of tpic is that the tpic
preprocessor itself is not readily available on most machines. It
is written in yacc and C language. It is mainly for UNIX or
similar system.

\eepic, as an extension to both \LaTeX\ and \epic, tries to
alleviate some of the limitations in \LaTeX, \epic\ and tpic by
generating tpic \verb|special|s using \TeX\ commands instead of
any preprocessor program, but at the same time provides
compatibility with the original commands such that when a DVI
driver which understands tpic \verb|special|s are not available,
the documents can still be formatted using standard \LaTeX\ and
\epic. However, the output probably will not be as good as
originally intended.

Currently, \eepic\ extends \LaTeX\ and \epic\ in the following
ways:
\begin{itemize}\parskip=0pt
\item Draws lines in any slopes.
\item Draws circles and discs (filled circle) in any radii.
\item Draws dotted lines and dash lines in a much faster way and
requires much less \TeX\ internal memory.
\item Provides more line thickness options.
\end{itemize}

Furthermore, \eepic\ introduces several new commands for:
\begin{itemize}\parskip=0pt
\item drawing of ellipsis and filled ellipsis
\item drawing of arcs
\item drawing of splines (cubic splines using control points)
\item drawing of polylines
\end{itemize}

All the affected commands in \LaTeX\ and \epic\ will be discussed
in the subsequent sections. The compatibility issues will be
described in the section~\ref{compat}.

In version 1.1, several bugs are fixed, and several commands for
area filled are added.
\clearpage


\section{Extension to \LaTeX}

In \LaTeX, drawing of lines and circles are done using special
fonts. Therefore only limited functions are provided. The
extensions in \eepic\ allow users to draw lines in any slope and
to draw circles in any sizes. However, the limitation of slopes
for vectors remains the same in the mean time. That is the slope
that can be handled is $\frac{x}{y}$ where $x$ and $y$ are
integers in the range $[-4,4]$. Please read \LaTeX\ manual for
details.


\subsection{\tt \BS line}
The syntax of \verb|\line| is the same as that in \LaTeX:
\begin{quote}
\verb|\line(|$x$,$y$\verb|){|{\em length}\verb|}|
\end{quote}
But now $x$ and $y$ can be any integer values within the limit of
\TeX. Furthermore, there is no more lower limit for {\em length}
parameter.


\subsection{\tt \BS circle}
The syntax of \verb|\circle| is the same as that in \LaTeX:
\begin{quote}
\verb|\circle{|{\em diameter}\verb|}|
\end{quote}
or
\begin{quote}
\verb|\circle*{|{\em diameter}\verb|}|
\end{quote}
But now the {\em diameter} parameter can be any number acceptable
by \TeX\ and a circle with the specified diameter (exactly) will
be drawn.


\subsection{\tt \BS oval}
The \verb|\oval| command is changed such that the maximum
diameter of the quarter circles at the corners can be set to any
values. This is done by setting the variable \verb|\maxovaldiam|
to the desire \TeX{} dimension. The default is 40pt.


\clearpage


\section{Extension to EPIC}
\epic\ is an enhancement to the Picture Environment of \LaTeX.
\epic\ generates standard DVI files and requires only standard
\LaTeX\ fonts. Some of the functions it provides are:
\begin{center}
\begin{tabular}{lll}
\verb|\multiputlist| & \verb|\dottedline| & \verb|\putfile| \\
\verb|\matrixput| & \verb|\dashline| \\
\verb|\grid| & \verb|\drawline| \\
\end{tabular}
\end{center}
Details can be found in the \epic\ manual. 

Extensions to \epic\ in \eepic\ include better line drawing
output, faster operation and less memory requirement. The
commands affected are:
\begin{enumerate}\parskip=0pt
\item \verb|\drawline|
\item \verb|\dashline|
\item \verb|\dottedline|
\end{enumerate}
And the three ``\verb|*join|'' environments are indirectly
affected also.


\subsection{\tt \BS drawline}
The syntax of \verb|\drawline| is:
\begin{quote}
\begin{alltt}
\BS{}drawline[{\em{}stretch}](\(x\sb{1}\),\(y\sb{1}\))(\(x\sb{2}\),\(y\sb{2}\))\ldots(\(x\sb{n}\),\(y\sb{n}\))
\end{alltt}
\end{quote}
where {\em stretch} is an integer between $-100$ and infinity.
However any number greater than 0 are the same. An negative {\em
stretch} in \verb|\drawline| will call \verb|\dashline|.

The thickness of the line is affected by \verb|\thinlines|,
\verb|\thicklines| and \verb|\Thicklines| declarations.
Horizontal and vertical lines are drawn using rules.


\subsection{\tt \BS dottedline}
The syntax of \verb|\dottedline| is:
\begin{quote}
\begin{alltt}
\BS{}dottedline[{\em{}dot character\/}]\{{\em{}dotgap\/}\}(\(x\sb{1}\),\(y\sb{1}\))(\(x\sb{2}\),\(y\sb{2}\))\ldots(\(x\sb{n}\),\(y\sb{n}\))
\end{alltt}
\end{quote}
where {\em dot character} is the character used in drawing the
``dotted'' line. {\em dotgap} is the interdot gap in terms of
\verb|\unitlength|. \verb|\special|s will only be generated if no
optional dot character is specified.

The size of the dots are affected by \verb|\thinlines|,
\verb|\thicklines| and \verb|\Thicklines| declarations.


\subsection{\tt \BS dashline}
The syntax of \verb|\dashline| is:
\begin{quote}
\begin{alltt}
\BS{}dashline[{\em{}stretch}]\{{\em{}dash-length}\}[{\em{}inter-dot-gap}](\(x\sb{1}\),\(y\sb{1}\))(\(x\sb{2}\),\(y\sb{2}\))\ldots(\(x\sb{n}\),\(y\sb{n}\))
\end{alltt}
\end{quote}
where {\em stretch} is an integer between $-100$ and infinity. If
{\em inter-dot-gap} is not specified, dashes are drawn in solid
lines, otherwise, dashes are drawn using dotted lines.

The thickness of the line is affected by \verb|\thinlines|,
\verb|\thicklines| and \verb|\Thicklines| declarations.


\clearpage


\section{New Commands}
\eepic\ introduces a number of new commands. Except the
\verb|\path| commands, all other new commands do not have any
equivalents in \LaTeX\ and \epic. Please read
section~\ref{compat} about the compatibility issues.


\subsection{\tt \BS allinethickness}
Set the line thickness of all line drawing commands including
lines in any slopes, circles, ellipsis, arcs, ovals and splines.
Note there are only two `l' in the command. The syntax is:
\begin{quote}
\verb|\allinethickness{|{\em dimension}\verb|}|.
\end{quote}


\subsection{\tt \BS Thicklines}
The syntax is:
\begin{quote}
\verb|\Thicklines|
\end{quote}
With the \verb|\Thicklines| declaration, thickness of lines drawn
will be about 1.5 times of \verb|\thicklines|.


\subsection{\tt \BS path}
\verb|\path| is a fast version of \verb|\drawline|. Optional {\em
stretch\/} argument is not allowed and so it always draw solid
lines. The syntax is:
\begin{quote}
\begin{alltt}
\BS{}path(\(x\sb{1}\),\(y\sb{1}\))(\(x\sb{2}\),\(y\sb{2}\))\ldots(\(x\sb{n}\),\(y\sb{n}\))
\end{alltt}
\end{quote}
\verb|\path| is mainly used in drawing complex paths.


\subsection{\tt \BS spline}
Syntax of \verb|\spline| is the same as \verb|\path|.
\begin{quote}
\begin{alltt}
\BS{}spline(\(x\sb{1}\),\(y\sb{1}\))(\(x\sb{2}\),\(y\sb{2}\))\ldots(\(x\sb{n}\),\(y\sb{n}\))
\end{alltt}
\end{quote}
\verb|\spline| draws an Chaikin's curve which passes through only
the first and last point. All other points are control points
only.


\subsection{\tt \BS ellipse}
The command \verb|\ellipse| draws an ellipse by specifying the
x-diameter and y-diameter.
\begin{quote}
\begin{alltt}
\BS{}ellipse\{{\em{}x-diameter}\}\{{\em{}y-diameter}\}
\end{alltt}
\end{quote}
or
\begin{quote}
\begin{alltt}
\BS{}ellipse*\{{\em{}x-diameter}\}\{{\em{}y-diameter}\}
\end{alltt}
\end{quote}
When {\em x-diameter} is equal to {\em y-diameter}, the command
is equivalent to \verb|\circle| or \verb|\circle*|.


\subsection{\tt \BS arc}
\verb|\arc| draws an circular arc. The syntax is
\begin{quote}
\begin{alltt}
\BS{}arc\{{\em{}diameter}\}\{{\em{}start-angle}\}\{{\em{}end-angle}\}
\end{alltt}
\end{quote}
{\em diameter} is specified in \verb|\unitlength| and both {\em
start-angle} and {\em end-angle} are in radian. {\em start-angle}
must be within 0 and $2\pi$ and {\em end-angle} can be any value
between {\em start-angle} and {\em start-angle} + $2\pi$. Arcs
are drawn in clockwise direction with angle 0 pointing to the
right on the paper.


\subsection{\tt \BS filltype\{....\}}
The command specifies the type of area fill for \verb|\circle*| and
\verb|\ellipse*|. The command itself does not draw anything. 
It only changes the interpretation of \verb|*| in the two commands
specified above.
The syntax of the command is:
\begin{quote}
\begin{alltt}
\BS{}filltype\{{\em{}area-fill-type}\}
\end{alltt}
\end{quote}
The legal area fill type are:
\begin{itemize}
\item black (default)
\item white
\item shade
\end{itemize}

For example, to change area fill type to white fill, the
following command should be used.
\begin{verbatim}
        \filltype{white}
\end{verbatim}


These commands are only intended for advance users (those who
know what they are doing). They are included mainly because
\verb|fig2epic|\footnote{Another program written by me to convert
Fig output file to eepic format.} generate these commands. The
commands are:
\begin{center}
\begin{tabular}{|l|p{4.4in}|}\hline
\multicolumn{1}{|c|}{commands} & \qquad Description \\ \hline
\verb|\blacken| & Black fill the next figure \\
\verb|\whiten| & White fill the next figure \\
\verb|\shade| & Shade the interior next figure \\
\verb|\texture| & Specify the pattern used for the next
\verb|shade| command. The pattern will remain effective until it is
changed by another \verb|\texture| command.
The syntax is:
\begin{quote}
\begin{alltt}
\BS{}texture\{{\em 32 32-bit hexadecimal numbers}\}
\end{alltt}
\end{quote}
An example (the default) is:
\begin{quote}
\begin{verbatim}
\texture{cccccccc 0 0 0 cccccccc 0 0 0 
         cccccccc 0 0 0 cccccccc 0 0 0 
         cccccccc 0 0 0 cccccccc 0 0 0 
         cccccccc 0 0 0 cccccccc 0 0 0}
\end{verbatim} 
\end{quote} \\
\hline
\end{tabular}
\end{center}
The exact interpretation of the above commands are probably
device driver dependent. I did most of tests using \verb|iptex|
(imagen1) and several tests using \verb|dvips|. 
The description below may not apply to other
device drivers.

The commands that can be specified after \verb|\blacken|,
\verb|\whiten| and \verb|\shade| include \verb|\path|,
\verb|\circle| (without \verb|*|), \verb|\ellipse| (again without
\verb|*|) and \verb|\arc|. The drawings do not have to be closed.
The imagen printer will automatically draw an imaginary line from
the starting point to the end point, and then fill the figure.
When using \verb|iptex|, the outline of the figures are drawn but not
in \verb|dvips|. In another words, when using \verb|iptex|, the command:
\begin{quote}
\begin{verbatim}
\shade\circle{10}
\end{verbatim}
\end{quote}
will draw a circle will the circumference in solid line and the
interior is filled in the pattern active at that time. However, when using
dvips, the circle will not have the circumference drawn in sold line.

\clearpage


\section{Examples}
I shamelessly stole two examples from the \epic\ manual so that
you can compare the results. The third and fourth examples are
created by FIG and then converted to \eepic\ using
\verb|fig2epic| which is also written by me.

\unitlength=1mm


\subsection{Example 1}
\begin{figure}[h]
\begin{center}
%\newcommand{\plotchar}{\makebox(0,0){\large $\otimes$}}
\begin{picture}(100,100)(0,0)
\put(0,0){\tiny \grid(100,100)(5,5)[0,0]}
\drawline(10,5)(60,10)(85,20)(90,60)(100,95)
\drawline[-50](10,0)(65,5)(90,15)(95,55)
\thicklines
\dottedline{1.4}(10,10)(60,20)(75,35)(95,95)
\dashline{2}(80,90)(50,80)(30,50)(10,40)
\dashline{2}[0.5](80,80)(50,70)(30,40)(10,30)
\dashline[-30]{2}[0.5](80,70)(50,60)(30,30)(10,20)
\end{picture}
\end{center}
\caption[]{\normalsize An example of Various Line Drawing Commands}
\end{figure}
\clearpage


\subsection{Example 2}
\newcount\xjunk
\newcount\yjunk
\begin{figure}[h]
\begin{center}
\begin{tiny}
\begin{picture}(140,140)(-70,-70)
\thinlines
\xjunk=60 \yjunk=3
\loop
\drawline(0,0)(\xjunk,\yjunk)
\drawline(0,0)(-\xjunk,\yjunk)
\drawline(0,0)(\xjunk,-\yjunk)
\drawline(0,0)(-\xjunk,-\yjunk)
\put(\xjunk,\yjunk){\plotchar}
\put(-\xjunk,\yjunk){\plotchar}
\put(\xjunk,-\yjunk){\plotchar}
\put(-\xjunk,-\yjunk){\plotchar}
\put(\xjunk,\yjunk){\makebox(8,0)[l]{\ \ (\number\xjunk,\number\yjunk)}}
\put(-\xjunk,\yjunk){\makebox(-4,0)[r]{(\number-\xjunk,\number\yjunk)}}
\put(\xjunk,-\yjunk){\makebox(8,0)[l]{\ \ (\number\xjunk,\number-\yjunk)}}
\put(-\xjunk,-\yjunk){\makebox(-4,0)[r]{(\number-\xjunk,\number-\yjunk)}}
\ifnum\yjunk < 53 \advance\yjunk by 10 %will go till 53.
\repeat
\xjunk=3 \yjunk=60
\loop
\drawline(0,0)(\xjunk,\yjunk)
\drawline(0,0)(-\xjunk,\yjunk)
\drawline(0,0)(\xjunk,-\yjunk)
\drawline(0,0)(-\xjunk,-\yjunk)
\put(\xjunk,\yjunk){\plotchar}
\put(-\xjunk,\yjunk){\plotchar}
\put(\xjunk,-\yjunk){\plotchar}
\put(-\xjunk,-\yjunk){\plotchar}
\put(\xjunk,\yjunk){\makebox(0,7)[t]{(\number\xjunk,\number\yjunk)}}
\put(-\xjunk,\yjunk){\makebox(0,5)[t]{(\number-\xjunk,\number\yjunk)}}
\put(\xjunk,-\yjunk){\makebox(0,-4)[t]{(\number\xjunk,\number-\yjunk)}}
\put(-\xjunk,-\yjunk){\makebox(0,-6)[t]{(\number-\xjunk,\number-\yjunk)}}
\ifnum\xjunk < 53 \advance\xjunk by 10
\repeat
\end{picture}
\end{tiny}
\end{center}
\caption[]{\normalsize Test Sample: Lines of various slopes with
\verb|thinlines|}
\end{figure}
\clearpage


\subsection{Example 3}
\begin{figure}[htbp]
\hrule
\begin{center}
\setlength{\unitlength}{0.0125in}
\begin{picture}(444,125)(0,-10)
\thicklines
\drawline(304.318,26.338)(303.000,31.000)(301.969,26.267)
\put(311.808,31.269){\arc{17.624}{4.8481}{9.3942}}
\drawline(158.742,66.792)(161.000,63.000)(160.792,67.408)
\put(168.688,65.312){\arc{16.054}{2.8495}{7.4287}}
\drawline(143.367,53.233)(147.000,54.000)(143.433,55.033)
\put(147.250,60.750){\arc{13.509}{1.6078}{6.2462}}
\put(34,46){\oval(68,26)}
\put(163,46){\ellipse{22}{22}}
\put(231,46){\ellipse{22}{22}}
\put(299,46){\ellipse{22}{22}}
\put(366,46){\ellipse{22}{22}}
\put(433,46){\ellipse{22}{22}}
\drawline(73,46)(146,46)
\drawline(138.000,44.000)(146.000,46.000)(138.000,48.000)
\drawline(181,46)(214,46)
\drawline(206.000,44.000)(214.000,46.000)(206.000,48.000)
\drawline(247,46)(282,46)
\drawline(274.000,44.000)(282.000,46.000)(274.000,48.000)
\drawline(315,46)(349,46)
\drawline(341.000,44.000)(349.000,46.000)(341.000,48.000)
\drawline(383,46)(416,46)
\drawline(408.000,44.000)(416.000,46.000)(408.000,48.000)
\spline(294,34)
(254,4)(194,-1)(164,14)(159,29)
\drawline(163.427,22.043)(159.000,29.000)(159.632,20.778)
\spline(229,34)
(209,19)(184,19)(169,34)
\drawline(176.071,29.757)(169.000,34.000)(173.243,26.929)
\spline(221,35)
(199,29)(175,35)
\drawline(183.246,35.000)(175.000,35.000)(182.276,31.119)
\spline(354,59)
(294,79)(244,59)
\drawline(250.685,63.828)(244.000,59.000)(252.171,60.114)
\spline(359,64)
(318,92)(224,84)(179,55)
\drawline(184.641,61.015)(179.000,55.000)(186.808,57.652)
\put(390,52){\makebox(0,0)[lb]{\raisebox{0pt}[0pt][0pt]{\twltt C}}}
\put(323,50){\makebox(0,0)[lb]{\raisebox{0pt}[0pt][0pt]{\twltt B}}}
\put(298,94){\makebox(0,0)[lb]{\raisebox{0pt}[0pt][0pt]{\twltt B}}}
\put(270,74){\makebox(0,0)[lb]{\raisebox{0pt}[0pt][0pt]{\twltt A}}}
\put(321,16){\makebox(0,0)[lb]{\raisebox{0pt}[0pt][0pt]{\twltt A}}}
\put(260,18){\makebox(0,0)[lb]{\raisebox{0pt}[0pt][0pt]{\twltt C}}}
\put(258,51){\makebox(0,0)[lb]{\raisebox{0pt}[0pt][0pt]{\twltt A}}}
\put(221,16){\makebox(0,0)[lb]{\raisebox{0pt}[0pt][0pt]{\twltt C}}}
\put(196,35){\makebox(0,0)[lb]{\raisebox{0pt}[0pt][0pt]{\twltt B}}}
\put(192,50){\makebox(0,0)[lb]{\raisebox{0pt}[0pt][0pt]{\twltt A}}}
\put(167,77){\makebox(0,0)[lb]{\raisebox{0pt}[0pt][0pt]{\twltt C}}}
\put(129,64){\makebox(0,0)[lb]{\raisebox{0pt}[0pt][0pt]{\twltt B}}}
\put(19,42){\makebox(0,0)[lb]{\raisebox{0pt}[0pt][0pt]{\twltt Start}}}
\put(162,42){\makebox(0,0)[lb]{\raisebox{0pt}[0pt][0pt]{\twltt 1}}}
\put(228,42){\makebox(0,0)[lb]{\raisebox{0pt}[0pt][0pt]{\twltt 2}}}
\put(298,42){\makebox(0,0)[lb]{\raisebox{0pt}[0pt][0pt]{\twltt 3}}}
\put(363,42){\makebox(0,0)[lb]{\raisebox{0pt}[0pt][0pt]{\twltt 4}}}
\put(432,42){\makebox(0,0)[lb]{\raisebox{0pt}[0pt][0pt]{\twltt *}}}
\end{picture}

\end{center}
\caption{The finite automaton to detect occurrences of $P$=`$AABC$'.}
\medskip
\hrule
\end{figure}


\subsection{Example 4}
\begin{figure}[htbp]
\hrule
\begin{center}
\setlength{\unitlength}{0.0125in}
\begin{picture}(264,218)(0,-10)
\thinlines
\put(125,137){\ellipse{10}{10}}
\put(160,92){\ellipse{10}{10}}
\put(20,112){\ellipse{10}{10}}
\put(80,20){\ellipse{10}{10}}
\put(90,92){\ellipse{10}{10}}
\put(174,20){\ellipse{10}{10}}
\put(245,117){\ellipse{10}{10}}
\put(170,177){\ellipse{10}{10}}
\put(80,177){\ellipse{10}{10}}
\dashline{4.000}(173,31)(163,82)
\drawline(166.502,74.534)(163.000,82.000)(162.577,73.765)
\dashline{4.000}(88,85)(81,28)
\drawline(79.990,36.184)(81.000,28.000)(83.960,35.697)
\dashline{4.000}(164,20)(89,20)
\drawline(97.000,22.000)(89.000,20.000)(97.000,18.000)
\dashline{4.000}(180,27)(239,107)
\drawline(235.861,99.374)(239.000,107.000)(232.642,101.749)
\dashline{4.000}(151,90)(101,90)
\drawline(109.000,92.000)(101.000,90.000)(109.000,88.000)
\dashline{4.000}(95,99)(118,128)
\drawline(114.596,120.489)(118.000,128.000)(111.462,122.975)
\dashline{4.000}(154,100)(130,128)
\drawline(136.725,123.228)(130.000,128.000)(133.688,120.624)
\dashline{4.000}(119,143)(87,171)
\drawline(94.338,167.237)(87.000,171.000)(91.704,164.227)
\drawline(24,104)(73,28)
\drawline(66.984,33.640)(73.000,28.000)(70.346,35.807)
\drawline(166,168)(134,143)
\drawline(139.073,149.501)(134.000,143.000)(141.535,146.349)
\drawline(29,110)(80,95)
\drawline(71.761,95.339)(80.000,95.000)(72.889,99.176)
\drawline(237,112)(171,93)
\drawline(178.134,97.135)(171.000,93.000)(179.241,93.291)
\drawline(178,173)(239,124)
\drawline(231.511,127.451)(239.000,124.000)(234.016,130.569)
\drawline(88,177)(160,177)
\drawline(152.000,175.000)(160.000,177.000)(152.000,179.000)
\drawline(75,171)(27,119)
\drawline(30.957,126.235)(27.000,119.000)(33.896,123.522)
\put(194,107){\makebox(0,0)[lb]{\raisebox{0pt}[0pt][0pt]{\twltt 7}}}
\put(47,109){\makebox(0,0)[lb]{\raisebox{0pt}[0pt][0pt]{\twltt 6}}}
\put(33,61){\makebox(0,0)[lb]{\raisebox{0pt}[0pt][0pt]{\twltt 5}}}
\put(156,149){\makebox(0,0)[lb]{\raisebox{0pt}[0pt][0pt]{\twltt 4}}}
\put(207,156){\makebox(0,0)[lb]{\raisebox{0pt}[0pt][0pt]{\twltt 3}}}
\put(39,151){\makebox(0,0)[lb]{\raisebox{0pt}[0pt][0pt]{\twltt 2}}}
\put(121,180){\makebox(0,0)[lb]{\raisebox{0pt}[0pt][0pt]{\twltt 1}}}
\put(85,102){\makebox(0,0)[lb]{\raisebox{0pt}[0pt][0pt]{\twltt I}}}
\put(159,102){\makebox(0,0)[lb]{\raisebox{0pt}[0pt][0pt]{\twltt H}}}
\put(122,148){\makebox(0,0)[lb]{\raisebox{0pt}[0pt][0pt]{\twltt G}}}
\put(0,105){\makebox(0,0)[lb]{\raisebox{0pt}[0pt][0pt]{\twltt F}}}
\put(76,0){\makebox(0,0)[lb]{\raisebox{0pt}[0pt][0pt]{\twltt E}}}
\put(171,1){\makebox(0,0)[lb]{\raisebox{0pt}[0pt][0pt]{\twltt D}}}
\put(256,115){\makebox(0,0)[lb]{\raisebox{0pt}[0pt][0pt]{\twltt C}}}
\put(168,187){\makebox(0,0)[lb]{\raisebox{0pt}[0pt][0pt]{\twltt B}}}
\put(74,186){\makebox(0,0)[lb]{\raisebox{0pt}[0pt][0pt]{\twltt A}}}
\end{picture}

\end{center}
\caption{Breath-first search beginning at A}
\medskip
\hrule
\end{figure}


\clearpage


\section{Bugs}
\begin{itemize}
\item The \verb|\circle*| and \verb|\ellipse*| may not work on
all DVI drivers especially some previewers. If you have any
problem, you should remove the related code in eepic.sty and use
the \LaTeX{} \verb|\cirlce*| commands. To find the related codes,
search for \verb|\special{bk}| in the file.
\item The alignment of the quarter circles and the lines in
\verb|\oval| command may not be correct on all printers because
the command relies on the precise interpretation of the tpic
specials which are not defined clearly. If you have any problem,
you should either fixed that by changing the position of the
quarter circles or remove the whole extended \verb|\oval| command
from \verb|eepic.sty|.
\item The area fill commands probably will not work on most
previewers, and different device drivers may produce slightly
different results.
\end{itemize}


\section{Compatibility}\label{compat}
If you want your \TeX\ file to be compatible with \LaTeX\ and
\epic\ but at the same time you want a better print out by using
\eepic, you must avoid several features in \eepic.
\begin{itemize}
\item Try not to use \verb|\line| commands and use
\verb|\drawline| instead because \verb|\line| in \LaTeX\ only
supports a limited set of slope.
\item Do not use \verb|\arc| command. Use \verb|\spline| if a
curve is really necessary.
\item Avoid using solid or small inter-dot gap in drawing long
dash lines. They used up a lot of \TeX\ memory in original \epic.
You should use \verb|\drawline| with negative stretch to draw the
dash lines.
\end{itemize}

If you want to use any of the extended commands in \eepic, you
must include the \eepic\ emulation macros (\verb|eepicemu|) in
the file. The extended commands are emulated in the following
ways.
\begin{itemize}
\item Circles larger than 40pt will be drawn using \verb|\oval|.
\item Ellipsis will be drawn using \verb|\oval|.
\item Spline will be approximated by \verb|\drawline|.
\item \verb|\path| will be substituted by \verb|\drawline|.
\item \verb|\Thicklines| will be substituted by
\verb|\thicklines|.
\item \verb|\allinethickness| will be substituted by
\verb|\thicklines| and \verb|\linethickness|.
\end{itemize}

\clearpage


\section{Installation}
There are two possible ways of installing \eepic. If your DVI
printer driver program supports the tpic specials, you should use
the standard \verb|eepic.sty| file. If your DVI printer driver
does not supports the tpic specials or you want to generate a
standard DVI file without any special commands, you should use
the file \verb|eepicemu.sty|.

EPIC is available on \verb|cs.rochester.edu| and
\verb|sun.soe.clarkson.edu| for anonymous ftp and e-mail request.

\subsection{Use tpic Specials}
First of all, you have to put a copy of \verb|epic.sty| and
\verb|eepic.sty| in a place where \LaTeX\ can find it. See
section 4 of \epic\ manual for more information.

Both \epic\ and \eepic\ have been implemented as document style
options \verb|epic| and \verb|eepic|. When using
\verb|epic| and \verb|eepic|, \verb|eepic| must come
after \verb|epic| in the \verb|\documentstyle| command. For example:
\begin{quote}
\begin{verbatim}
\documentstyle[epic,eepic]{article}
\end{verbatim}
\end{quote}

If you only need the extended \LaTeX\ commands and the new
\eepic\ commands, you may include only \verb|eepic| in the
\verb|\documentstyle| command. But then the \eepic\ emulation
package will not work. I strongly recommend you to use
\eepic\ with \epic\ all the time.

\subsection{No tpic Specials}
If you want to get a standard DVI file but you need the extended
\eepic\ commands, you should rename \verb|eepicemu.sty| to
\verb|eepic.sty| and put that in a place where \LaTeX\ can find
it. Remember \verb|\arc| command is not supported and the output
will not be as good as standard \eepic. Furthermore, you cannot
use the emulation package with \LaTeX\ alone. You have to include
\verb|epic| also.

\end{document}

